% Appendix B

\chapter{Full Code Listings From Ch. 2} % Main appendix title

\label{AppendixB} % For referencing this appendix elsewhere, use \ref{AppendixA}

\lhead{Appendix B. \emph{Full Code Listings From Chapter 2 }} % This is for the header on each page - perhaps a shortened title

\section{Simulation \#4:  Simple Orbit}

\setstretch{1}
\begin{lstlisting}[breaklines=true, frame=single, numbers=left, caption=Simple Orbit Simulation, label=lst:orbit]
var canvas = document.getElementById('canvas');
var context = canvas.getContext('2d'); 
var canvas_bg = document.getElementById('canvas_bg');
var context_bg = canvas_bg.getContext('2d');

var planet;
var sun;
var m = 1; // planet's mass
var M = 1000000; // sun's mass
var G = 1;
var t0,dt;

window.onload = init; 

function init() {
  // create a stationary sun
  sun = new Ball(70,'orange',M);
  sun.pos2D = new Vector2D(275,200);  
  sun.draw(context_bg);
  // create a moving planet     
  planet = new Ball(10,'blue',m);
  planet.pos2D = new Vector2D(200,50);
  planet.velo2D = new Vector2D(80, -40);
  planet.draw(context);
  // make the planet orbit the sun
  t0 = new Date().getTime(); 
  animFrame();
};

function animFrame(){
  animId = requestAnimationFrame(animFrame,canvas);
  onTimer(); 
}
function onTimer(){
  var t1 = new Date().getTime(); 
  dt = 0.001*(t1-t0); 
  t0 = t1;  
  if (dt>0.1) {dt=0;};  
  move();
}
function move(){      
  moveObject(planet);
  calcForce();
  updateAccel();
  updateVelo(planet);
}

function moveObject(obj){
  obj.pos2D = obj.pos2D.addScaled(obj.velo2D,dt); 
  context.clearRect(0, 0, canvas.width, canvas.height);
  obj.draw(context);  
}
function calcForce(){
  force = Forces.gravity(G,M,m,planet.pos2D.subtract(sun.pos2D)); 
} 
function updateAccel(){
  acc = force.multiply(1/m);
} 
function updateVelo(obj){
  obj.velo2D = obj.velo2D.addScaled(acc,dt);        
}
\end{lstlisting}



\section{Simulation \#5:  Escape Velocity}
\begin{lstlisting}[breaklines=true, frame=single, numbers=left, caption=Escape Velocity Simulation]
var canvas = document.getElementById('canvas');
var context = canvas.getContext('2d'); 
var canvas_bg = document.getElementById('canvas_bg');
var context_bg = canvas_bg.getContext('2d');

var planet;
var sun;
var m = 1; // planet's mass
var M = 1000000; // sun's mass
var G = 1;
var t0,dt;
var i = 0;


window.onload = init; 

function init() {
  // create a stationary sun
  sun = new Ball(400,'orange',M);
  sun.pos2D = new Vector2D(500,2900); 
  sun.draw(context_bg);
  // create a moving planet     
  planet = new Ball(10,'blue',m);
  planet.pos2D = new Vector2D(500,2490);
  planet.velo2D = new Vector2D(0,-70);
  console.log(planet.velo2D.length())

  planet.draw(context);
  // make the planet orbit the sun
  t0 = new Date().getTime(); 
  animFrame();
};

function animFrame(){
  animId = requestAnimationFrame(animFrame,canvas);
  onTimer(); 
}
function onTimer(){
  var t1 = new Date().getTime(); 
  dt = 0.001*(t1-t0); 
  t0 = t1;  
  if (dt>0.1) {dt=0;};  
  move();
}
function move(){      
  moveObject(planet);
  calcForce();
  updateAccel();
  updateVelo(planet);
  i++;
  if(i%15 ===0){console.log(planet.velo2D.length());}
}

function moveObject(obj){
  obj.pos2D = obj.pos2D.addScaled(obj.velo2D,dt); 
  context.clearRect(0, 0, canvas.width, canvas.height);
  obj.draw(context);  
}
function calcForce(){
  force = Forces.gravity(G,M,m,planet.pos2D.subtract(sun.pos2D)); 
} 
function updateAccel(){
  acc = force.multiply(1/m);
} 
function updateVelo(obj){
  obj.velo2D = obj.velo2D.addScaled(acc,dt);        
}
\end{lstlisting}


\section{Simulation \#6:  Kepler's 2nd Law}
\begin{lstlisting}[breaklines=true, frame=single, numbers=left, caption=Kepler's 2nd Law Simulation]
var canvas = document.getElementById('canvas');
var context = canvas.getContext('2d'); 
var canvas_bg = document.getElementById('canvas_bg');
var context_bg = canvas_bg.getContext('2d');

var planet;
var sun;
var m = 1; // planet's mass
var M = 10000000; // sun's mass
var G = 1;
var t0,dt;
var i =0;

window.onload = init; 

function init() {
  // create a stationary sun
  sun = new Ball(70,'orange',M);
  sun.pos2D = new Vector2D(475,250);  
  sun.draw(context_bg);
  // create a moving planet     
  planet = new Ball(2,'blue',m);
  planet.pos2D = new Vector2D(180,270);
  planet.oldpos2D = new Vector2D(planet.x, planet.y)
  planet.velo2D = new Vector2D(100, -180);
  planet.draw(context);
  // make the planet orbit the sun
  t0 = new Date().getTime(); 
  animFrame();
};

function animFrame(){
  animId = requestAnimationFrame(animFrame,canvas);
  onTimer(); 
}
function onTimer(){
  var t1 = new Date().getTime(); 
  dt = 0.001*(t1-t0); 
  t0 = t1;  
  if (dt>0.1) {dt=0;};  
  move();
}
function move(){    
  i++;
  moveObject(planet);
  calcForce();
  updateAccel();
  updateVelo(planet);
  if(i<960){
    if(i%30===0){
      console.log(i);
      context.strokeStyle = `white';
      context.moveTo(planet.x, planet.y);
      context.lineTo(sun.x, sun.y);
      context.stroke();
      var dr = Vector2D.distance(planet.pos2D, planet.oldpos2D);
      var r =  Vector2D.distance(planet.pos2D, sun.pos2D);
      console.log('dA is equal to:  %f', .5*r*dr);
      planet.oldpos2D=planet.pos2D;
    }
  }
}

function moveObject(obj){
  obj.pos2D = obj.pos2D.addScaled(obj.velo2D,dt); 
  obj.draw(context);  
}
function calcForce(){
  force = Forces.gravity(G,M,m,planet.pos2D.subtract(sun.pos2D)); 
} 
function updateAccel(){
  acc = force.multiply(1/m);
} 
function updateVelo(obj){
  obj.velo2D = obj.velo2D.addScaled(acc,dt);        
}
\end{lstlisting}

