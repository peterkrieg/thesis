% Appendix A

\chapter{Full Code Listings From Ch. 1} % Main appendix title

\label{AppendixA} % For referencing this appendix elsewhere, use \ref{AppendixA}

\lhead{Appendix A. \emph{Full Code Listings From Chapter 1 }} % This is for the header on each page - perhaps a shortened title

\section{Simulation \#1:  Simple Ball Bouncing}

\setstretch{1}
\begin{lstlisting}[breaklines=true, frame=single, numbers=left, caption=A basic ball bouncing simulation, label=lst:ballbounce1]
var canvas = document.getElementById(`canvas');
var context = canvas.getContext(`2d'); 

canvas.height = screen.height-200;
canvas.width = screen.width -100;

var radius = 20;
var color = ``red";
var g = .1635; // acceleration due to gravity
var x = 40;  // initial horizontal position
var y = 40;  // initial vertical position
var vx = parseFloat(prompt(`what is the initial horizontal speed of ball you would like?(recommended values of 1-20'));  // initial horizontal speed 
var vy = 0;  // initial vertical speed
 
window.onload = init; 
 
function init() {
  setInterval(onEachStep, 1000/60); // 60 fps
};
 
function onEachStep() {
  vy += g; // gravity increases the vertical speed
  x += vx; // horizontal speed increases horizontal position 
  y += vy; // vertical speed increases vertical position

  if (y > canvas.height - radius){ // if ball hits the ground
    y = canvas.height - radius; // reposition it at the ground
    vy *= -0.8; // then reverse and reduce its vertical speed
  }
  if (x > canvas.width - radius){ // if ball hits right wall
    x = canvas.width - radius; // reposition it right at wall 
    vx *= -0.8;  // then reduce and reverse horizontal speed
  }
  if (x < radius){  // if ball hits left wall
    x = radius;  // reposition it right at wall
    vx *= -0.8  // then reverse and reduce horizontal speed
  }
  drawBall(); // draw the ball
};
 
function drawBall() {
  with (context){
    clearRect(0, 0, canvas.width, canvas.height); 
    fillStyle = color;
    beginPath();
    arc(x, y, radius, 0, 2*Math.PI, true);
    closePath();
    fill();
  };
};
\end{lstlisting}

\section{Simulation \#2: More Advanced Ball Bouncing}

\begin{lstlisting}[breaklines=true, frame=single, numbers=left, caption=More Advanced Ball Bouncing Simulation, label=lst:ballbounce2]
var canvas = document.getElementById('canvas');
var context = canvas.getContext('2d');

canvas.height=screen.height-300;
canvas.width=screen.width-100;

var x = 40;
var y =40;
var vy = 0;
var ay = 0;
var m = 1;
var r = 20;
var rSI = r* 0.0002309090909;  // radius in SI, converting px to m
var C_r = .8;  // Coefficient of restitution (tennis ball would be .8)
var rho = 1.2;    // density of air would be 1.2, water would be 1000
var dt = 60/1000;  // Time Step
var C_d = 0.47; //Coefficient of drag for sphere
var A = Math.PI * rSI * rSI;
var color = 'red';

window.onload = init();
  
function init(){
  console.log(vy);
  setInterval(onEachStep, 1000/60);
}

function onEachStep(){ 
  var fy = 0;
  fy += m * 9.81;   // weight force
  if (vy>=0){
    fy -= 1* 0.5 *rho * C_d *A *vy *vy; 
  } 
  else {
    fy += 1*0.5 *rho *C_d *A *vy *vy;
  }

  ay = fy / m;
  vy += ay * dt;
  y += vy;
  
  // simple collision detection for floor only
  if (y + r > canvas.height){ 
    vy *= -C_r; 
    y = canvas.height - r;  
  }
  drawBall();
}

function drawBall() {
  with (context){
    clearRect(0, 0, canvas.width, canvas.height); 
    fillStyle = color;
    beginPath();
    arc(x, y, r, 0, 2*Math.PI, true);
    closePath();
    fill();
  };
};
\end{lstlisting}

\section{Simulation \#3: Multiple Balls Bouncing}

\begin{lstlisting}[breaklines=true, frame=single, numbers=left, caption=Multiple Balls Bouncing Simulation, label=lst:ballsbouncing]


var canvas = document.getElementById('canvas');
var context = canvas.getContext('2d'); 

canvas.height = screen.height-200;
canvas.width = screen.width -100;

var g = 0.1635;
var balls;
var numBalls = prompt('how many balls would you like to have bounce?'); 
var C_d = .8;
 
window.onload = init; 
 
function init() {
	balls = []; // creates empty array
	for (var i=0; i<numBalls; i++){
		radius = Math.random()*20+5;
		var ball = new Ball();	
		ball.x = 50;
		ball.y = 75;
		ball.radius =  radius;
		ball.vx = Math.random()*15;
		ball.vy = (Math.random()-0.5)*10;
		ball.color = getRandomColor();
		ball.draw(context);
		balls.push(ball);
	}  
	setInterval(onEachStep, 1000/60); // 60 fps
};
 
function onEachStep() {
	context.clearRect(0, 0, canvas.width, canvas.height); 
	for (var i=0; i<numBalls; i++){
		var ball = balls[i];
		ball.vy += g;     

		if (ball.vx >0){   // while vx is still positive, decrease it incrementally to represent air resistance/friction
    ball.vx -= .001;
  } else{
    ball.vx === 0;   // the instant vx is 0 or negative, it is set to 0 to stop the movement in x direction
  }
		ball.x += ball.vx; 
		ball.y += ball.vy; 
			
		if (ball.y > canvas.height - ball.radius){ 
			ball.y = canvas.height - ball.radius; 
			ball.vy *= -C_d; 
		}
		if (ball.x + ball.radius > canvas.width){
			ball.x = canvas.width - ball.radius; 
			ball.vx *= -C_d;
		}
		if (ball.x < ball.radius){
			ball.x = ball.radius;
			ball.vx *= -C_d;
		}
		ball.draw(context); 
	} 
};

function getRandomColor() {
    var letters = '0123456789ABCDEF'.split('');
    var color = '#';
    for (var i = 0; i < 6; i++ ) {
        color += letters[Math.floor(Math.random() * 16)];
    }
    return color;
}

\end{lstlisting}

\vspace{2cm}

\begin{lstlisting}[breaklines=true, frame=single, numbers=left, caption=Ball.js file used for prototype ball object label=lst:ballprototype]
function Ball (radius, color) {
  this.radius = radius;
  this.color = color;
  this.x = 0;
  this.y = 0;
  this.vx = 0;
  this.vy = 0;
}

Ball.prototype.draw = function (context) {
	context.fillStyle = this.color;
  context.beginPath();
  context.arc(this.x, this.y, this.radius, 0, 2*Math.PI, true);
  context.closePath();
  context.fill();  
};
\end{lstlisting}

