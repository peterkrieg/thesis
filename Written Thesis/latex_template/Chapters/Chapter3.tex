% Chapter 3

\chapter{Rigid Body Motion} % Main chapter title

\label{Chapter3} % For referencing the chapter elsewhere, use \ref{Chapter1} 

\lhead{Chapter 3. \emph{Rigid Body Motion}} % This is for the header on each page - perhaps a shortened title

%----------------------------------------------------------------------------------------

Previous chapters have disregarded rotational motion of solid bodies.  This chapters examines the mechanics of rigid body rotations and collisions through angular momentum fixed axis rotation.  Simulations will involve rigid body collisions.


\section{Background Physics}

Previous sections in this thesis have simply involved translational motion.  The theorem of rigid body motion states that the displacement of any rigid body can be decomposed into two independent motions: the translation of the center of mass, and the rotation about the center of mass.  A rigid body in general is defined as an object that maintains its shape and size when a force is applied to it.  In reality, all objects experience some level of deformation, however for the purpose of these simulations it is safe to ignore this.    


