% Introduction

\chapter*{INtroduction}
\label{intro} % For referencing the chapter elsewhere, use \ref{Chapter1} 

\lhead{Introduction \emph{Presenting the canvas, etc}} % This is for the header on each page - perhaps a shortened title

%----------------------------------------------------------------------------------------

\section{Welcome and Thank You}
Welcome to this \LaTeX{} Thesis Template, a beautiful and easy to use template for writing a thesis using the \LaTeX{} typesetting system.


%----------------------------------------------------------------------------------------

\section{Learning \LaTeX{}}

\LaTeX{} is not a WYSIWYG (What You See is What You Get) program, unlike word processors such as Microsoft Word or Apple's Pages. Instead, a document written for \LaTeX{} is actually a simple, plain text file that contains \emph{no formatting}. You tell \LaTeX{} how you want the formatting in the finished document by writing in simple commands amongst the text, for example, if I want to use \textit{italic text for emphasis}, I write the `$\backslash$\texttt{textit}\{\}' command and put the text I want in italics in between the curly braces. This means that \LaTeX{} is a ``mark-up'' language, very much like HTML.

\subsection{A (not so short) Introduction to \LaTeX{}}

If you are new to \LaTeX{}, there is a very good eBook -- freely available online as a PDF file -- called, ``The Not So Short Introduction to \LaTeX{}''. The book's title is typically shortened to just ``lshort''. You can download the latest version (as it is occasionally updated) from here:\\
\href{http://www.ctan.org/tex-archive/info/lshort/english/lshort.pdf}{\texttt{http://www.ctan.org/tex-archive/info/lshort/english/lshort.pdf}}

It is also available in several other languages. Find yours from the list on this page:\\
\href{http://www.ctan.org/tex-archive/info/lshort/}{\texttt{http://www.ctan.org/tex-archive/info/lshort/}}

It is recommended to take a little time out to learn how to use \LaTeX{} by creating several, small `test' documents. Making the effort now means you're not stuck learning the system when what you \emph{really} need to be doing is writing your thesis.

\subsection{A Short Math Guide for \LaTeX{}}

If you are writing a technical or mathematical thesis, then you may want to read the document by the AMS (American Mathematical Society) called, ``A Short Math Guide for \LaTeX{}''. It can be found online here:\\
\href{http://www.ams.org/tex/amslatex.html}{\texttt{http://www.ams.org/tex/amslatex.html}}\\
under the ``Additional Documentation'' section towards the bottom of the page.

\subsection{Common \LaTeX{} Math Symbols}
There are a multitude of mathematical symbols available for \LaTeX{} and it would take a great effort to learn the commands for them all. The most common ones you are likely to use are shown on this page:\\
\href{http://www.sunilpatel.co.uk/latexsymbols.html}{\texttt{http://www.sunilpatel.co.uk/latexsymbols.html}}

You can use this page as a reference or crib sheet, the symbols are rendered as large, high quality images so you can quickly find the \LaTeX{} command for the symbol you need.

\subsection{\LaTeX{} on a Mac}
 
The \LaTeX{} package is available for many systems including Windows, Linux and Mac OS X. The package for OS X is called MacTeX and it contains all the applications you need -- bundled together and pre-customised -- for a fully working \LaTeX{} environment and workflow.
 
MacTeX includes a dedicated \LaTeX{} IDE (Integrated Development Environment) called ``TeXShop'' for writing your `\texttt{.tex}' files and ``BibDesk'': a program to manage your references and create your bibliography section just as easily as managing songs and creating playlists in iTunes.

%----------------------------------------------------------------------------------------

