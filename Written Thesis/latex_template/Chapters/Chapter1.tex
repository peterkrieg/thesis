% Chapter 1

\chapter{Some Basic Simulations} % Main chapter title

\label{Chapter1} % For referencing the chapter elsewhere, use \ref{Chapter1} 

\lhead{Chapter 1. \emph{Chapter Title Here}} % This is for the header on each page - perhaps a shortened title

%----------------------------------------------------------------------------------------


While the introduction outlined the computer programming necessary to produce simulations in general, this chapter will start to deal with the physics necessary to make simulations seem realistic.  In this chapter, I will outline some examples of simulations with balls bouncing, and discussing the basic mechanics involved through the code.



\section{Welcome and Thank You}
Welcome to this \LaTeX{} Thesis Template, a beautiful and easy to use template for writing a thesis using the \LaTeX{} typesetting system.

If you are writing a thesis (or will be in the future) and its subject is technical or mathematical (though it doesn't have to be), then creating it in \LaTeX{} is highly recommended as a way to make sure you can just get down to the essential writing without having to worry over formatting or wasting time arguing with your word processor.

\LaTeX{} is easily able to professionally typeset documents that run to hundreds or thousands of pages long. With simple mark-up commands, it automatically sets out the table of contents, margins, page headers and footers and keeps the formatting consistent and beautiful. One of its main strengths is the way it can easily typeset mathematics, even \emph{heavy} mathematics. Even if those equations are the most horribly twisted and most difficult mathematical problems that can only be solved on a super-computer, you can at least count on \LaTeX{} to make them look stunning.




\begin{equation} \label{eq:blah}
x = x^2 = 5
\end{equation}

as you can see in equation \ref{eq:blah} this is interesting stuff!!


\begin{equation} \label{eq:blah2}
\frac{x}{y} = \frac{222}{\frac{\frac{45}{33}}{20}}
\end{equation}

as you can see in  \ref{eq:blah2}  alks;dkglals;kdglka;lsdg!!!!


\setstretch{1.2}
\begin{lstlisting}[breaklines=true, frame=single, numbers=left, caption=Some Code, label=lst:examplecode]
var canvas = document.getElementById('canvas');
var context = canvas.getContext('2d'); 

canvas.height = screen.height-200;
canvas.width = screen.width -100;

var radius = 20;
var color = "red";
var g = .15; // acceleration due to gravity
var x = 50;  // initial horizontal position
var y = 50;  // initial vertical position
var vx = 2;  // initial horizontal speed
var vy = 0;  // initial vertical speed
 
window.onload = init; 

function init() {
  setInterval(onEachStep, 1000/60); // 60 fps
};
 
function onEachStep() {
  vy += g; // gravity increases the vertical speed

  if (vx >0){   // while vx is still positive, decrease it incrementally to represent air resistance/friction
    vx -= .001;
  } else{
    vx === 0;   // the instant vx is 0 or negative, it is set to 0 to stop the movement in x direction
  }

  x += vx; // horizontal speed increases horizontal position 
  y += vy; // vertical speed increases vertical position
 
  if (y > canvas.height - radius){ // if ball hits the ground
    y = canvas.height - radius; // reposition it at the ground
    vy *= -0.9; // then reverse and reduce its vertical speed
  }
  if (x > canvas.width + radius){ // if ball goes beyond canvas
    x = -radius; // wrap it around 
  }
  drawBall(); // draw the ball
};
\end{lstlisting}
\setstretch{2}
As you can see in listing \ref{lst:examplecode}  stuff got interesting!!  \cite{lamport94}






















