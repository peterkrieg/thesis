% Conclusion

\chapter{Conclusion} % Main chapter title

\label{Conclusion} % For referencing the chapter elsewhere, use \ref{Chapter1} 

\lhead{Conclusion} % This is for the header on each page - perhaps a shortened title

%----------------------------------------------------------------------------------------


Physics simulations can be made realistic through JavaScript.  This thesis only contains a few of the many possibilities.  The methods of programming in this thesis can be summarized to a few simple steps:

\begin{enumerate}  
\item Initialize global variables and constants
\item Initialize the repetitive loop of animation, in a variety of different ways
\item Prepare conditions to be updated on each frame of animation, such as position, speed, acceleration or force
\item Draw the current conditions to the canvas, and then erase the canvas to give the illusion of motion
\end{enumerate} 

By providing specific rules of code for each frame of the simulation, the program can emulate realistic physics processes.   




These simulations can be made much more advanced and complex if they are incorporated in three dimensions.  This can be done by utilizing web GL, a JavaScript API which is used to render 3D objects in the canvas.  


































