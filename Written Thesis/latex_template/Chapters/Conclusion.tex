% Conclusion

\chapter{Conclusion} % Main chapter title

\label{Conclusion} % For referencing the chapter elsewhere, use \ref{Chapter1} 

\lhead{Conclusion} % This is for the header on each page - perhaps a shortened title

%----------------------------------------------------------------------------------------

\setstretch{2}
Physics simulations can be made highly realistic through JavaScript.  This thesis only presents a few of the many possibilities.  The methods of programming applied in this thesis can be summarized in a few simple steps:

\begin{enumerate}  
\item Initialize global variables and constants
\item Initialize the repetitive loop of animation, in a variety of different ways
\item Prepare conditions to be updated on each frame of animation, such as position, speed, acceleration or force
\item Draw the current conditions to the canvas, and then erase the canvas to give the illusion of motion
\end{enumerate} 

By providing specific rules of code for each frame of the simulation, the program can emulate realistic physics processes.  The simulations can easily be edited to allow for many different situations.  This is the main advantage of simulations as opposed to video animations that can't be edited easily once created.     


These simulations can be made more advanced and complex if they are incorporated in three dimensions.  This can be done by utilizing web GL, a JavaScript API which is used to render 3D objects in the canvas.  If I had more time or did a full-year thesis, I would look to expand these simulations into more advanced 3D simulations.  Much of the physics of chapter 3 doesn't take into account the vector $\vec{k}$ simply because the simulations in this thesis only involved $\vec{i}$ and $\vec{k}$.  


The most challenging part of this thesis was overcoming small javascript bugs that occurred.  I had to invest considerable time at the beginning of the thesis to sharpen my JavaScript skills.  Sometimes the code confused me when using multiple different functions, based from different prototypes.  Chapter 3 was also challenging for me to apply the mechanics concepts to actual code.  Overall, this thesis was a lot of work investigating physics through a different lens, but I enjoyed it a lot.


































